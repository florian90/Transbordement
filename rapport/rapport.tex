\documentclass[a4paper, 12pt]{report}

\usepackage[utf8]{inputenc}
\usepackage[T1]{fontenc}
\usepackage[francais]{babel}
% \usepackage{geomery}

\title{AG41 - Transbordement}
\author{Staine Florian}
\date\today

\setlength{\hoffset}{-18pt}        
\setlength{\oddsidemargin}{0pt} 	% Marge gauche sur pages impaires
\setlength{\evensidemargin}{9pt} 	% Marge gauche sur pages paires
\setlength{\marginparwidth}{54pt} 	% Largeur de note dans la marge
\setlength{\textwidth}{485pt} 		% Largeur de la zone de texte (17cm)
\setlength{\voffset}{-18pt} 		% Bon pour DOS
\setlength{\marginparsep}{7pt} 		% Séparation de la marge
\setlength{\topmargin}{0pt} 		% Pas de marge en haut
\setlength{\headheight}{5pt} 		% Haut de page
\setlength{\headsep}{5pt} 		% Entre le haut de page et le texte
\setlength{\footskip}{27pt} 		% Bas de page + séparation
\setlength{\textheight}{708pt} 		% Hauteur de la zone de texte (25cm)

\begin{document}
\maketitle
\parindent=0em

\chapter{Objectif}
L'objectif de ce projet est de résoudre un problème d'optimisation exacte, 
dérivé d'un problème de transport. 
\section{Problème}
En plus de devoir transiter d'un fournisseur à un client, 
les ressources doivent passer par une plateforme de transbordement.

\paragraph{Chaque arc possède : }
\begin{itemize}
 \item une capacité maximale
 \item un coût fixe ainsi qu'un coût unitaire d'utilisation
 \item un temps de transport
\end{itemize}

\paragraph{Chaque plateforme possède : }
\begin{itemize}
 \item un coût unitaire de transbordement
 \item un temps de transbordement\newline
\end{itemize}

Le coût total est constitué de la somme de plusieurs coûts :\\
En effet, si un arc transporte un ou plusieurs produit, son coût fixe sera pris en compte, 
en plus du coût unitaire multiplié par le nombre de produits transportés.
Pour chaque produit transitant par une plateforme, un coût unitaire de transbordement sera ajouté au coût total.\\

Nous avons aussi d'autres contraintes que dans le problème initial.\\
Tout d'abord, les arc ont tous une capacité maximale, ce qui signifie que 
qu'il ne peut transiter plus de produit que la capacité maximale. 
Les arc sont utilisés au maximum une fois, ils ne peut pas y avoir des aller-retour 
pour transporter plus de ressources.\\
Un temps maximal de transport devra être respecté pour le transport de toutes les ressources.
De plus, tous les produits partent au même moment et sont transportés en parallèles. 
Cela signifie que pour chaque cheminement emprunté pour atteindre les clients,
les temps de transport plus le temps de transbordement doivent être inférieurs au temps maximum autorisé.

\chapter{Modèle mathématique}
\section{Paramètres}
$n$ : nombre total de nœuds\\
$n_f$ : nombre de nœuds fournisseurs\\
$n_p$ : nombre de nœuds plates-formes\\
$n_c$ : nombre de nœuds clients \newline

$b_i$ : demande ou disponibilité du nœud\\
$g_i$ : Coût unitaire de transbordement\\
$s_i$ : Temps de transbordement\newline

$u_{i,j}$ : Capacité de l'arc $(i,j)$\\
$c_{i,j}$ : Coût fixe de l'arc $(i,j)$\\
$h_{i,j}$ : Coût unitaire de l'arc $(i,j)$\\
$t_{i,j}$ : Temps de transport de l'arc $(i,j)$

\section{Variables}
$x_{i,j}$ : nombre de produit transportés par l'arc $(i,j)$\\
$y_{i,j,k}$ : Nombre de produits transportés par la route $(i,j)\ (j,k)$\\

\section{Objectif}
$ min\ z = \sum_{i=1,\ j=1,\ x_{i,j}\not=0\ }^{n_f,\ n_p}
	(x_{i,j} \cdot h_{i,j} + c_{i,j} + x_{i,j} \cdot g_j) 
	    + \sum_{j=1,\ k=1,\ x{i,j} \not=0}^{n_p,\ n_c}(x_{j,k} \cdot h_{j,k} + c_{j,k})$

\section{Contraintes}
$ (C1):\ \forall (i,j)\in n^2,\ x_{i,j} \leq u_{i,j} $ \\
$ (C2):\ \forall i\in n, \ \sum_{j=1}^{n}(x_{i,j} - x_{j,i}) = -b_i$\\
$ (C3):\ \forall (i,j)\in n^2,\ \sum_{k=1}^{n_c} y_{i,j,k} + \sum_{k=1}^{n_f} y_{k,j,i} = x_{i,j}$\\
$ (C4):\ \forall (i,j,k)\in n^3,\ y_{i,j,k} \not= 0,\ 
	t_{i,j} + t_{j,k} + s_j \leq T$\\

\chapter{Résolution algorithmique}
\section{Algorithme utilisé}

Pour résoudre ce problème, un algorithme de type Branch and Bound sera utilisé.
Cela consiste à séparer en deux l'ensemble des solutions selon s'il ont fait un choix ou son contraire.
Certains choix sont évités s'ils possèdent une borne minimal de l'objectif plus grand qu'une solution déjà existante. 

\section{Problèmes à résoudre}
En premier lieu il est nécessaire de calculer une première solution. 
Cela permet d'éviter toutes les solutions plus mauvaises que cette première.
Plus cette première solution sera proche de la solution finale, 
plus il sera facile de converger rapidement vers la solution optimale car un plus grand nombre de choix seront évités.
\newline{}
\newline{}
Ensuite, une borne minimum la plus élevée possible doit être définie 
pour permettre d'éliminer les solution moins efficace que la solution déjà existante. 
Tout le temps de résolution du problème est dépendant de ces deux algorithmes, 
qui sont le choix d'une première solution efficace et le calcul d'une borne inférieure. 
\newline{}
\newline{}
Ensuite, il est possible d'éliminer certains choix lorsqu'il n'est pas possible de résoudre le problème 
à partir d'un sous-ensemble en solutions, cela permet encore un gain de temps non négligeable dans la résolution.

\section{Méthodes utilisées}
\subsection{Solution initiale}
Il n'y a pas à proprement parler de solution initiale qui est trouvée grâce à une méthode différente.\\
Cependant, une sélection différente est faite sur les arcs selon si on a déjà trouvé une solution initiale ou pas.\\
En effet, les chemins sont choisis de manière à maximiser le nombre de ressources transportés, 
ce qui permet de converger plus rapidement vers une solution initiale, même si celle-ci n'est pas d'excellente qualité.\\

Même si le principe de l'algorithme est évidement de couper le plus haut possible dans l'arbre des solutions 
pour éviter de perdre du temps dans des solutions loin de l'optimum, 
l'ordre des choix n'a pas d'incidence sur la meilleur solution trouvée 
du fait que toutes les possibilités sont envisagés puis testés.\\

Lorsqu'une solution initiale à été trouvée, l'algorithme se concentre sur la recherche des branches les moins chères
permettant de se diriger vers la solution optimale.

\subsection{Borne minimale}
La borne minimale est calculé en additionnant le cout de la solution partielle ainsi que le cout minimum de transport pour les nœuds qui ne sont pas encore vides. \\
Pour estimer le cout des nœuds, les arcs sont tillés suivant leur coût de transport. 
Ensuite, on prends les chemins par ordre du moins chère au plus chère
jusqu'à ce que toutes les ressources soient transportées. 

\subsection{Solutions déduites}
Le principe du Branch and Bound est de réduire l'ensemble des solutions en limitant la liberté sur certaines variables. \\
Ici, la recherche de chemins est limité par une liste de chemins évités, chemins que l'on ne peut pas emprunter.\\
Lors de la recherche de chemins, si tous les chemins évités ne  peuvent pas être sélectionnés. 

\subsection{Bugs connus, limites du programme et problèmes rencontrés}
Lors de la recherche de chemin à emprunter, toutes les possibilités ne sont pas testés.
En effet, seul la valeur maximale d'utilisation d'un chemin est testé, 
et si celle ci fait partie de la liste des chemins évités, elle est retirée. \\
Cependant, il est possible qu'un même chemin avec une quantité transportée inférieure
soit utilisable mais ne soit pas considérée dans les chemins possibles.\\
En considérant aussi ceux-ci, le temps de calcul est souvent beaucoup plus long et ne permettait,
dans un grand nombre de cas, même pas d'obtenir une solution initiale. \\ \hbox

L'algorithme actuellement en place est souvent trop lent pour permettre de trouver des solutions 
aux problèmes de taille supérieur à 50 nœuds. 
Cela vient en partie de la relativement faible efficacité de l'élimination des solutions, 
ce qui conduit à l'évaluation d'un grand nombre de possibilités

\end{document}
